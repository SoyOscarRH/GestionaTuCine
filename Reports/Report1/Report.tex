% ****************************************************************************************
% ************************      NAME OF DOCUMENT      ************************************
% ****************************************************************************************

% =======================================================
% =======         HEADER FOR DOCUMENT        ============
% =======================================================
    % *********   DOCUMENT ITSELF   **************
    \documentclass[11pt, fleqn]{article}                             %Type of docuemtn and size of font and left eq
    \usepackage[margin=1.2in]{geometry}                             %Margins and Geometry pacakge
    \usepackage{ifthen}                                             %Allow simple programming
    \usepackage{hyperref,pdfpages}                                           %Create MetaData for a PDF and LINKS!
    \hypersetup{pageanchor=false}                                   %Solve 'double page 1' warnings in build
    \setlength{\parindent}{0pt}                                     %Eliminate ugly indentation
    \author{Oscar Andrés Rosas}                                     %Who I am

    % *********   LANGUAJE AND UFT-8   *********
    \usepackage[spanish]{babel}                                     %Please use spanish
    \usepackage[utf8]{inputenc}                                     %Please use spanish - UFT
    \usepackage[T1]{fontenc}                                        %Please use spanish
    \usepackage{textcmds}                                           %Allow us to use quoutes
    \usepackage{changepage}                                         %Allow us to use identate paragraphs
    \usepackage{lipsum}                                             %Allow to put dummy text

    % *********   MATH AND HIS STYLE  *********
    \usepackage{ntheorem, amsmath, amssymb, amsfonts}               %All fucking math, I want all!
    \usepackage{mathrsfs, mathtools, empheq}                        %All fucking math, I want all!
    \usepackage{centernot}                                          %Allow me to negate a symbol
    \decimalpoint                                                   %Use decimal point

    % *********   GRAPHICS AND IMAGES *********
    \usepackage{graphicx}                                           %Allow to create graphics
    \usepackage{wrapfig}                                            %Allow to create images
    \graphicspath{ {Graphics/} }                                    %Where are the images :D

    % *********   LISTS AND TABLES ***********
    \usepackage{listings}                                           %We will be using code here
    \usepackage[inline]{enumitem}                                   %We will need to enumarate
    \usepackage{tasks}                                              %Horizontal lists
    \usepackage{longtable}                                          %Lets make tables awesome
    \usepackage{booktabs}                                           %Lets make tables awesome
    \usepackage{tabularx, array}                                    %Lets make tables awesome
    \usepackage{multirow}                                           %Lets make tables awesome
    \usepackage{multicol}                                           %Create multicolumns

    % *********   HEADERS AND FOOTERS ********
    \usepackage{fancyhdr}                                           %Lets make awesome headers/footers
    \pagestyle{fancy}                                               %Lets make awesome headers/footers
    \setlength{\headheight}{16pt}                                   %Top line
    \setlength{\parskip}{0.5em}                                     %Top line
    \renewcommand{\footrulewidth}{0.5pt}                            %Bottom line

    \lhead{                                                         %Left Header
        \hyperlink{section.\arabic{section}}                        %Make a link to the current chapter
        {\normalsize{\textsc{\nouppercase{\leftmark}}}}             %And fot it put the name
    }

    \rhead{                                                         %Right Header
        \hyperlink{section.\arabic{section}.\arabic{subsection}}    %Make a link to the current chapter
            {\footnotesize{\textsc{\nouppercase{\rightmark}}}}      %And fot it put the name
    }

    \rfoot{\textsc{\small{Primer Borrador :D}}}                     %This will always be a footer  

    \fancyfoot[L]{                                                  %Algoritm for a changing footer
        \ifthenelse{\isodd{\value{page}}}                           %IF ODD PAGE:
                {\footnotesize                                      %Send the page
                    {\textsc{Bases de Datos}}}                      %Send the page
                {\footnotesize                                      %Send the author
                    {\textsc{2CM12}}}                               %Send the author
    }
    
    
    
% ========================================
% ===========   COMMANDS    ==============
% ========================================

    % =====  GENERAL TEXT  ==========
    \newcommand \Quote {\qq}                                        %Use: \Quote to use quotes
    \newcommand \Over {\overline}                                   %Use: \Bar to use just for short
    \newcommand \ForceNewLine {$\Space$\\}                          %Use it in theorems for example
    
    \newenvironment{Indentation}[1][0.75em]                         %Use: \begin{Inde...}[Num]...\end{Inde...}
    {\begin{adjustwidth}{#1}{}}                                     %If you dont put nothing i will use 0.75 em
    {\end{adjustwidth}}                                             %This indentate a paragraph
    \newenvironment{SmallIndentation}[1][0.75em]                    %Use: The same that we upper one, just 
    {\begin{adjustwidth}{#1}{}\begin{footnotesize}}                 %footnotesize size of letter by default
    {\end{footnotesize}\end{adjustwidth}}                           %that's it
        
    % =====  GENERAL MATH  ==========
    \DeclareMathOperator \Space {\quad}                             %Use: \Space for a cool mega space
    \DeclareMathOperator \MiniSpace {\;}                            %Use: \Space for a cool mini space
    \newcommand \Such {\MiniSpace|\MiniSpace}                       %Use: \Such like in sets
    \newcommand \Also {\Space \text{y} \Space}                      %Use: \Also so it's look cool
    \newcommand \Remember[1]{\Space\text{\scriptsize{#1}}}          %Use: \Remember so it's look cool

    \newtheorem{Theorem}{Teorema}[section]                          %Use: \begin{Theorem}[Name]\label{Nombre}...
    \newtheorem{Corollary}{Colorario}[Theorem]                      %Use: \begin{Corollary}[Name]\label{Nombre}...
    \newtheorem{Lemma}[Theorem]{Lemma}                              %Use: \begin{Lemma}[Name]\label{Nombre}...
    \newtheorem{Definition}{Definición}[section]                    %Use: \begin{Definition}[Name]\label{Nombre}...

    \newcommand{\Set}[1]{\left\{ \MiniSpace #1 \MiniSpace \right\}} %Use: \Set {Info}
    \newcommand{\Brackets}[1]{\left[ #1 \right]}                    %Use: \Brackets {Info} 
    \newcommand{\Wrap}[1]{\left( #1 \right)}                        %Use: \Wrap {Info} 
    \newcommand{\pfrac}[2]{\Wrap{\dfrac{#1}{#2}}}                   %Use: Put fractions in parentesis

    \newenvironment{MultiLineEquation}[1]                           %Use: To create MultiLine equations
        {\begin{equation}\begin{alignedat}{#1}}                     %Use: \begin{Multi..}{Num. de Columnas}
        {\end{alignedat}\end{equation}}                             %And.. that's it!
    \newenvironment{MultiLineEquation*}[1]                          %Use: To create MultiLine equations
        {\begin{equation*}\begin{alignedat}{#1}}                    %Use: \begin{Multi..}{Num. de Columnas}
        {\end{alignedat}\end{equation*}}                            %And.. that's it!


    % =====  LOGIC  ==================
    \DeclareMathOperator \doublearrow {\leftrightarrow}             %Use: \doublearrow for a double arrow
    \newcommand \lequal {\MiniSpace \Leftrightarrow \MiniSpace}     %Use: \lequal for a double arrow
    \newcommand \linfire {\MiniSpace \Rightarrow \MiniSpace}        %Use: \lequal for a double arrow
    \newcommand \longto {\longrightarrow}                           %Use: \longto for a long arrow

    % =====  NUMBER THEORY  ==========
    \DeclareMathOperator \Naturals  {\mathbb{N}}                     %Use: \Naturals por Notation
    \DeclareMathOperator \Primes    {\mathbb{P}}                     %Use: \Naturals por Notation
    \DeclareMathOperator \Integers  {\mathbb{Z}}                     %Use: \Integers por Notation
    \DeclareMathOperator \Racionals {\mathbb{Q}}                     %Use: \Racionals por Notation
    \DeclareMathOperator \Reals     {\mathbb{R}}                     %Use: \Reals por Notation
    \DeclareMathOperator \Complexs  {\mathbb{C}}                     %Use: \Complex por Notation

    % === LINEAL ALGEBRA & VECTORS ===
    \DeclareMathOperator \LinealTransformation {\mathcal{T}}        %Use: \LinealTransformation for a cool T
    \newcommand{\Mag}[1]{\left| #1 \right|}                         %Use: \Mag {Info} 

    \newcommand{\pVector}[1]{                                       %Use: \pVector {Matrix Notation} use parentesis
        \ensuremath{\begin{pmatrix}#1\end{pmatrix}}                 %Example: \pVector{a\\b\\c} or \pVector{a&b&c} 
    }
    \newcommand{\lVector}[1]{                                       %Use: \lVector {Matrix Notation} use a abs 
        \ensuremath{\begin{vmatrix}#1\end{vmatrix}}                 %Example: \lVector{a\\b\\c} or \lVector{a&b&c} 
    }
    \newcommand{\bVector}[1]{                                       %Use: \bVector {Matrix Notation} use a brackets 
        \ensuremath{\begin{bmatrix}#1\end{bmatrix}}                 %Example: \bVector{a\\b\\c} or \bVector{a&b&c} 
    }
    \newcommand{\Vector}[1]{                                        %Use: \Vector {Matrix Notation} no parentesis
        \ensuremath{\begin{matrix}#1\end{matrix}}                   %Example: \Vector{a\\b\\c} or \Vector{a&b&c}
    }

    % MATRIX
    \makeatletter                                                   %Example: \begin{matrix}[cc|c]
    \renewcommand*\env@matrix[1][*\c@MaxMatrixCols c] {             %WTF! IS THIS
        \hskip -\arraycolsep                                        %WTF! IS THIS
        \let\@ifnextchar\new@ifnextchar                             %WTF! IS THIS
        \array{#1}                                                  %WTF! IS THIS
    }                                                               %WTF! IS THIS
    \makeatother                                                    %WTF! IS THIS

    % TRIGONOMETRIC FUNCTIONS
    \newcommand{\Cos}[1]{\cos\Wrap{#1}}                             %Simple wrappers
    \newcommand{\Sin}[1]{\sin\Wrap{#1}}                             %Simple wrappers

    % === COMPLEX ANALYSIS ===
    \newcommand \Cis[1]  {\Cos{#1} + i \Sin{#1}}                    %Use: \Cis for cos(x) + i sin(x)
    \newcommand \pCis[1] {\Wrap{\Cis{#1}}}                          %Use: \pCis for the same ut parantesis
    \newcommand \bCis[1] {\Brackets{\Cis{#1}}}                      %Use: \bCis for the same to Brackets




% =====================================================
% ============        COVER PAGE       ================
% =====================================================
\begin{document}
\begin{titlepage}

    \center
    % ============ UNIVERSITY NAME AND DATA =========
    \textsc{\Large Bases de Datos}\\[0.5cm] 
    \textsc{\large 2CM12}\\[1.5cm]

    % ============ NAME OF THE DOCUMENT  ============
    \rule{\linewidth}{0.5mm} \\[1.0cm]
        { \huge \bfseries Primer Avance del Proyecto Final}\\[1.0cm] 
    \rule{\linewidth}{0.5mm} \\[2.0cm]
     
    % ============  MY INFORMATION  =================
    \begin{minipage}{0.4\textwidth}
        \begin{flushleft} \large
            \textbf{\textsc{Alumnos:}}\\
            \small{
                Maya Rocha Luis Emmanuel        \\
                Rosas Hernández Óscar Andrés    \\
                Dominguez Lopez Humberto        \\
                Hernández Ruiz Rafael
            }
        \end{flushleft}
    \end{minipage}
    ~
    \begin{minipage}{0.4\textwidth}
        \begin{flushright} \large
            \textbf{\textsc{Profesor: }}\\
            Euler Hernández Contreras.
        \end{flushright}
    \end{minipage}\\[3,5cm]


    \vfill

\end{titlepage}




% ======================================================================================
% ==================================     DOCUMENT  =====================================
% ======================================================================================


\section{Introducción}




    Como proyecto final de Bases de Datos nos disponemos a crear un Sistema basado en Web 
    (usando sobretodo PHP y el framework MaterializeCSS) para poder gestionar un Cine:
    \begin{itemize}
        \item Películas
        \item Horarios
        \item Dulceria
        \item Empleados
        \item Organización en general
    \end{itemize}


    Toda la información del sistema será almacenada en una base de datos creada con MySQL
    y que se comunicará con nuestra aplicación web usando las funciones de PHP con MySQL.

    \clearpage

    \begin{figure}[h!]
        \centering
        \includegraphics[width=0.80\textwidth]{Examples0.png}
        \caption{Boceto de FrontEnd de la aplicación web}
    \end{figure}

    \begin{figure}[h!]
        \centering
        \includegraphics[width=0.80\textwidth]{Examples1.png}
        \caption{Boceto de FrontEnd de la aplicación web}
    \end{figure}



\clearpage
\section{Problemática}
    
    Buscamos crear un sitema que permita organizar toda la administración de un Cine y todas
    sus subdivisiones de manera consisa, confiable y segura.

    Queremos que dicha aplicación web pueda ser usando con gran facilidad, importando lo menos posible
    el sistema operativo que usen nuestros clientes para usar la aplicación web así como el
    disposivito que utilicen, sea un tablet, un celular o una computadora (de cualquier gana)

    Buscamos que sea de fácil mejoramiento y que podamos actualizar y corregir de la manera más rápida
    cualquier error que tuviera nuestra aplicación web.

    Buscamos que la información que tenga nuestra aplicación web en la base de datos sea de fácil acceso
    para la creación posterior de nuevas interfaces y aplicaciones así como para su mantenimiento
    y escalabilidad.

    Buscamos la mayor seguridad en los datos, de tal manera que sean lo mas confiables posible.

    Buscamos que nuestra aplicación web permita diferentes perfiles de usuarios, para mejorar la confiabilidad
    y seguridad del sistema permitiendo a cierto tipo de usuarios solo hacer ciertas actividades y 
    tener la capacidad de leer / modificar ciertos datos dentro del sistema.



    \begin{figure}[h!]
        \centering
        \includegraphics[width=0.80\textwidth]{Examples2.png}
        \caption{Boceto de FrontEnd de la aplicación web}
    \end{figure}





\clearpage

    \section{Objetivo General:}

        Desarrollar un sistema que permita gestionar varios de los procesos necesarios en un cine;
        como son la dulcería, la venta de boletos y los roles de empleados.


    \section{Objetivos Particulares:}
    
        \begin{itemize}
            \item A través del desarrollo de un sistema gestionar los horarios de entrada y
                salida de los empleados en las diferentes áreas del cine.
            
            \item El sistema podrá gestionar los horarios y salas de las diferentes exhibiciones.

            \item El sistema gestionara la venta de boletos para las diferentes exhibiciones,
                además controlara la disponibilidad de boletos existentes para cada exhibición.

            \item El sistema controlara la dulcería, como la venta de los diferentes artículos,
                la existencia de estos y los proveedores.

            \item Se permitirá la creación de usuarios para restringir el acceso a la escritura y
                lectura de los diferentes datos, según el puesto que tenga el usuario en el cine.

        \end{itemize}


\section{Reglas de Negocio}

    \begin{tabular}{r ||c |m{7em} | m{15em} |c |c }
       &  ID & Nombre & Descripción & Prioridad & Origen \\ [0.5ex] 
       \hline\hline
      
        & RN1   & Control de Empleados                  &
        Un gerente que lleva el control de los empleado(vendedores de duces boletos y los que limpian)
        & Alta  & Propuestos\\

        & RN2   & Pagos                                 &
        El gerente se encarga de el pago a los proveedores
        & Media  & Propuestos\\

        & RN3   & Control de Salas                      &
        El gerente se encargar de determinar las salas para una pelicula y los horarios
        & Media  & Propuestos\\


        & RN4   & Control de Dulcería                   &
        Se debe de poder seleccionar los dulces y bebidas disponibles por ese día, 
        se venden 5 tipos diferentes de combos, el cliente debe elegir cual de los
        5 desea o puede no comprar ninguno.
        & Media  & Propuestos\\

        & RN4   & Control de Capacidad                   &
        Existen 6 salas, las cuales a su vez tienen 30 butacas, por lo que debe controlarse el numero
        de personas que entran.

        Al tratarse de un cine popular, solo se exhiben peliculas nuevas, y las tarifas de licencia
        para exhibir películas, las cuales pueden ser muy altas, sobre todo para películas importantes
        de estreno (puedes contratar a agentes de cine para ayudar con el proceso de conseguir las
        películas y la aprobación para exhibirlas)
        & Baja  & Propuestos\\

    \end{tabular}


\clearpage
\section{Requerimientos Básicos}

    \begin{tabular}{r ||c |m{7em} | m{15em} |c |c }
       &  ID & Nombre & Descripción & Prioridad & Origen \\ [0.5ex] 
       \hline\hline
      
        & RB1   & Seguridad                         &
        Todas las conexiones con la base de datos, así como la confiabilidad de los datos
        tienen que ser altas, para una mayor seguridad.
        & Alta  & Propuestos\\

        & RB2   & Compatibilidad de Navegadores     &
        El sistema tiene que ser accesible desde los navegadores mas usados
        & Baja  & Propuestos\\

        & RB3   & Compatibilidad de Plataformas     &
        El sistema puede ser usado desde diversas plataformas
        & Baja  & Propuestos\\

        & RB4   & Escalabilidad                     &
        La estructura tanto de la aplicación web como de la base de datos a nivel conceptual tiene
        que permitir el mantenimiento y la escalabilidad el proyecto.
        & Media  & Propuestos\\

        & RB4   & Roles                             &
        La aplicación web permite la existencia de distintos tipos de usuarios, con cada tipo
        las acciones disponibles cambian.
        & Media  & Propuestos\\

    \end{tabular}





\section{Requerimientos Funcionales}
\begin{tabular}{r ||c |c | m{15em} |c }
   &  ID & Nombre & Descripción & Origen \\ [0.5ex] 
   \hline\hline
  
    & RF1 & Gestionar Sala          &
    El Sistema permitirá al usuario gerente establecer los horarios y peliculas para una sala
    & RN2\\
    & RF2 & Gestionar Empleados     & 
    El sistema permitirá establecer que actividad llevará a cabo cada empleado impliendole
    realizar otra que no sea la indicada
    & RN1\\
    & RF3 & Control de Suministros  &
    El sistema podrá llevar stock de todos los productos utilizados en venta o en mantenimiento del cine
    & RN3\\
    & RF4 & Control de Entradas     &
    El sistema deberá indica el estatus de una sala al realizar la venta de un boleto en ella
    & RN6\\
    & RF5 & Gestión de Dulcería     & 
    El sistema permitira manejar las transacciones de venta en el departamento de dulceria
    & RN5\\
\end{tabular}



\section{Requerimientos No Funcionales}
    \begin{tabular}{r ||c |m{8em} | m{18em} |c }
       &   & Nombre & Descripción & Origen \\ [0.5ex] 
       \hline\hline
        & RNF1 & FrameWork &
        Sistema Desarrollado en: JQuery, MaterializeCSS
        & RB3\\

        & RNF2 & Entorno de Trabajo de Aplicaciones &
        PHP, HTML, Javascript, CSS, XAMPP(Apache)  
        & RB2\\

        & RNF3 & Bases de Datos & MySQL Server
        & RB1\\

        & RNF4 & Seguridad & Confidencia e integridad de los datos
        & RB1\\
    \end{tabular}


\clearpage
    \includepdf[scale=0.8,pages=1,pagecommand=\section{Diagrama de Contexto}]{Diagrama}



\clearpage
\section{Análisis de Riesgos}
    \includepdf[scale=0.8,pages=-]{Riesgos}











\end{document}
